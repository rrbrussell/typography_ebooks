\chapter{Punctuation}
\textsc{Punctuation} is a modern art. The ancients were entirely unaquainted
with the use of our commas, colons, \&c.; and wrote not only without any
distinction of members and periods, but also without distinction of words: which
custom continued till the year 360 before Christ. How the ancients read their
works, written in this manner, it is not easy to conceive. After the practice of
joining words together had ceased, notes of distinction were placed at the end
of every word. This practice, with some variation, continued a considerable
time. As it appears that the present usage of stops did not take place whilst
manuscripts and monumental inscriptions were the only known methods of conveying
knowledge, we must conclude that it was introduced with the art of printing.

The elements of Punctuation---regarded either as a branch of Grammar, or as a
science by itself---ought to be understood, in a greater or less degree, by
every one that wishes to obtain a correct knowledge of his native tongue. The
utility of Points may be appreciated by any person who attentively observes the
construction of sentences, their various conexions, and mutual dependence.
Without the use of Points, innumerable difficulties and ambiguities would
perplex the common reader, not merely in a want of knowledge where to pause, but
in the still greater defect of being unable to understand the meaning of his
author. Without them, indeed, the beauties of peoetry would be greatly obscured,
and the principles of philosophy little understood. The perusal of a single page
will bear testimony to its real importance. Scarcely a sentence can be read with
pleasure or interest, unless correctly pointed. The well-known speech of Norval,
in the tragedy \emph{Douglas}, may be read in such a manner as to make it appear
the most confused and preposterous composition that ever issued from the press.
Who can read the following lines, and deny that every young man should devote a
portion of his leisure time to the acquisition of so valuable an art?

\begin{quote}
My name is Norval on the Grampian hills\\
My father feeds his flock a frugal swain, \&c.\\
We fought and conquer'd ere a sword was drawn, \&c.
\end{quote}

The insertion of pauses will restore these passages to their natural beauty.

\begin{quote}
My name is Norval. On the Grampian hills\\
My father feeds his flock; a frugal swain,\\
Whose constant cares were to increase his store, \&c.\\
We fought and conquer'd. Ere a sword was drawn,\\
An arrow from my bow had pierc'd their chief, \&c.
\end{quote}

Notwithstanding the great utility of Punctuation, it has not received that
attention which its importance demands. It has been said, no doubt justly, that
all speaking, all writing, and all printing, would be very imperfect, without
the use of points. Considered either as the plaything of the pedant, or the
employment of the printer, it is equally neglected by the painter, the engraver,
the author, and the gentleman. Hence, the best workmanship of the painter and
the engraver, in which Punctuation is necessary, is imperfect by a want of
pauses, or ridiculous by their being misplaced. Indeed, it does not unfrequently
happen, that some of our most popular writers put their manuscripts into the
printer's hands, either altogether destitute of marks of Punctuation, or so
badly pointed as to create an unnecessary loss of time to the compositor.

But, although it has been thought, that this study is the peculiar province of
the compositor and the corrector, who might therefore be expected to be
qualified to perform their task with propriety; yet it will be readily admitted,
that many books have issued from the press, which, though unrivalled for
elegance of style, accuracy of orthography, and beauty of printing, are very
deficient in Punctuation.

To make these general observations, however, without granting many exceptions,
would partake more of the petulance of presumption, than of the candour of true
criticism. There are many masterpieces of \emph{composition}, in which both the
author and printer have executed their respective parts of the work with equal
elegance and propriety.

It has been objected to the study of this valuable art, that it is not subject
to any fixed principles. Where one author or printer uses a comma, another would
insert a semicolon, and \emph{vice versa}; and where one thinks a semicolon
ought to be employed, another prefers a colon. One pedagogue embarrasses the
learner with aditional point (the semicomma,) by giving it ``a local habitation
and a name;'' while another, ``all as frantic,'' discards the colon altogether,
as a useless pause.

It has been well stated in a recent publication on Punct\-uation---

\small
``That the art of Punctuation is not more varied or less certain, in its
character, than that of Composition; and that its \emph{essential} principles
are as fixed and determinate as those canons in syntax, which, though sometimes
violated by our best authors, are universally acknowledged to be indisputable.
Diversities in the application of these principles will no more prove, that
modes of pointing sentences are altogether arbitrary, than diversities in
styles of composition will demonstrate, that the labours of grammarians to
ascertain the laws of language must go for nought, and that every writer may
take whatever liberties he chooses, in opposition to reputable
usage.''\footnote{\textsc{A Treatise on Grammatical Puncuation}: Designed for
Authors, Printers, and Correctors of the Press; and for the use of Academies and
Schools. By John Wilson, Manchester: Printed and Published by the Author.}

\normalsize
The compilers of this little work would suggest to the young compositor
the propriety of acquiring the knowledge of an art, which is so advantageous to
himself, and so beneficial to the literary community. If, in commencing to set
types, he regard with indifference or inattention the subject of the writer, and
point only at random, he will acquire a habit of carelessness, the consequences
of which will be hurtful to his purse, and pernicious to his reputation.

The following Rules are laid before him, with a view of exhibiting the
principles of a branch in printing which has been to frequently subject to the
caprice of authors and correctors of the press; and which, consequently, has
been the means of creating, in the mind of the beginner, an antipathy to the
profession altogther. Although not entirely exempt from the technicalities of
grammar, these Rules will probably be found as plain and concise as the nature
of subject permits. The examples are, in connection with the Rules, of the
utmost consequence, as tending to illustration and perspicuity, by the proper
insertion of the pauses. But it is not enough, that the learner simply commits
them to memory: he must habituate himself to the pointing of epistles, essays,
or any other written composition; and compare his labours with the definitions
and observations herein given. By this means, his taste will be improved; and he
will be in the fair way of becoming---what should be the aim of every artist---a
good workman.

In confirmation of the above remarks we may add the following examples,
illustrative of the importance of Punctuation:---
