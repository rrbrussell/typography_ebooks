\documentclass[twoside,letterpaper]{book}
\usepackage{hyperref}

\begin{document}
\title{The Compositor's Text-Book:}
\author{John Graham}
\date{1848}
\maketitle
\tableofcontents

\phantomsection
\addcontentsline{toc}{section}{Preface}
\section*{Preface}
As it is usual for authors to introduce themselves to the notice of the public,
by expressing their motives for publishing, and explaining the nature and
tendency of their work; by tacitly commending it, in appealing to the judgement
of the reader; or by submitting it, with the greatest humility, to his candour;
it might be thought singular, if the writers of the following pages were not to
adopt a similar practice.

It is, however, almost unnecessary to make any apology for the present
undertaking. The large works of Printing are not only high-priced, but scarce;
while the smaller ones are very defective and incorrect: besides, a common fault
in all, is the omission of Rules on Punctuation; or, at least, the few
observations that are made on the subject, are so vague and imperfect, that the
young compositor takes it for granted that Punctuation cannot be fixed by any
principles whatever. This little book is intended to supply the deficiency, but
not to supersede the use, of Stower's Grammar, or any other valuable work on the
art of printing. The groundwork of the Rules contained herein will be found to
belong to Lindley Murray; while occasional assistance has been derived from the
English Grammar of Messrs. Angus and Churchill. The Rules, however, are
simplified as much as possible; new ones are added; and a greater variety of
appropriate examples are introduced, to illustrate what has been laid down.
A few observations are interspersed, to explain still farther the directions and
examples, to present to the learner the different views of writers or printers
on the subject, and to subject some practical hints which may be of utility to
the compositor. A short Essay is prefixed, to excite a desire of attaining a
knowledge of this much-despised, but useful art.

In enumerating  the advantages of this work, the writers do not mean to assert,
that the science of Pointing has now received all the light which can be thrown
upon it; but they humbly conceive, that the art of typography may be made more
easy by a careful, studied perusal of that part of the publication which relates
to this subject.

The remarks on Distributing and Composing are partly derived from the larger
works on printing, and are partly the result of the compilers' own experience.
A variety of Tables of Imposition is added, besides a selection of other useful
matter.

It would be a want of gratitude in the writers to conclude, without expressing
their thanks to those gentlemen who kindly pointed out several errors, which would
otherwirse have passed unnoticed

JOhn Wilson.
John Graham.

\end{document}
